\documentclass[a4paper,11pt]{article}
\usepackage{latexsym}
\usepackage[polish]{babel}
\usepackage{ucs}
\usepackage[utf8]{inputenc}
\usepackage[MeX]{polski}

\author{Sławomir Belter $\quad$ Patryk Żółtowski}

\title{MIABSR  \\ Laboratorium \\ Projekt DYNA}

\begin{document}

\maketitle

\section{Raport z projektu}

\subsection{Temat zadania }

Napisać autonomicznego agenta dla gry dyna bluster w technologii CORBA.

\subsection{Implementacja}

Projekt składa się zasadniczo z trzech głównych części:
\begin{itemize}
\item modułu do planowania trasy \emph{ARFF}
\item modułu do zbierania wiedzy na temat obecnego stanu gry
\item modułu do wyboru celu i taktyki
\end{itemize}

Planowanie trasy zostało zaimplementowane w oparciu o algorytm \emph{A*}. W
każdym przebiegu wyliczana jest najkrótsza trasa do wybranego celu. Podczas
planowania trasy brane są pod uwagę koszty przejścia przez dane węzły. Koszty te
są albo statyczne dla niektórych pól (np. dla ściany) lub wyliczane dynamiczne
w zależności od stanu gry.

Kolejny moduł odpowiada za zbieranie informacji o stanie gry takie jak
pojawiające się i zbierane bonusy, charakterystyki przeciwników, stan planszy.
W szczególności moduł analizuje pola na mapie i modyfikuje koszty węzłów w
siatce nawigacyjnej w ten sposób, że przedwcześnie detonuje postawione bomby. W
ten sposób bot uwzględnia większy koszt przejścia przez strefę wybuchu podczas
planowania trasy.

Ostatnia główna cześć projektu wykorzystuje wcześniejsze moduły do wyboru
strategii i celu. W pierwszej kolejności priorytetem jest ucieczka z
zagrożonych pozycji, a dopiero potem zbieranie bonusów. Na ofiarę wybierany jest
gracz, który znajduję się najbliżej. Bomby są stawiane gdy przeciwnik jest w
polu rażenia, pod warunkiem, że nie jesteśmy na zagrożonej pozycji.

Głowna klasa bota to: \emph{ai.bot.Aibot}. Domyślna nazwa bota: \emph{Bocik}

\end{document}


